\documentclass[10pt,a4paper]{report}

\usepackage[utf8]{inputenc}
\usepackage{array}

\usepackage{courier}

\begin{document}

\begin{table}[htbp]
    \begin{center}
      \begin{tabular}{l|rr|rr|rr}
  \hline
  Matrix                          & \multicolumn{4}{c}{spLLT}        & \multicolumn{2}{c}{MA87}                                         \\
  \hline
                                  & \multicolumn{2}{c}{OpenMP (gnu)} & \multicolumn{2}{c}{StarPU} & \multicolumn{2}{c}{MA87}            \\ 
  \cline{2-7}
                                  & nb                               & facto (s)                  & nb   & facto (s) & nb  & facto (s)  \\
  \hline
  Schmid/thermal2                 & 512                              & 1.801                      & 1024 & 2.123     & 256 & \bf 0.376  \\
  Rothberg/gearbox                & 256                              & \bf 0.220                  & 384  & 0.318     & 256 & 0.252      \\
  DNVS/m\_t1                      & 256                              & 0.205                      & 384  & 0.262     & 256 & \bf 0.194  \\
  Boeing/pwtk                     & 384                              & 0.241                      & 512  & 0.382     & 256 & \bf 0.235  \\
  Chen/pkustk13                   & 256                              & 0.222                      & 256  & 0.311     & 256 & \bf 0.220  \\
  GHS\_psdef/crankseg\_1          & 256                              & 0.254                      & 256  & 0.272     & 256 & \bf 0.242  \\
  Rothberg/cfd2                   & 256                              & 0.269                      & 384  & 0.350     & 256 & \bf 0.260  \\
  DNVS/thread                     & 256                              & \bf 0.203                  & 384  & 0.240     & 256 & 0.213      \\
  DNVS/shipsec8                   & 256                              & \bf 0.245                  & 384  & 0.330     & 256 & 0.250      \\
  DNVS/shipsec1                   & 256                              & \bf 0.247                  & 384  & 0.363     & 256 & 0.259      \\
  GHS\_psdef/crankseg\_2          & 256                              & 0.267                      & 384  & 0.310     & 256 & \bf 0.257  \\
  DNVS/fcondp2                    & 256                              & 0.334                      & 384  & 0.461     & 256 & \bf 0.271  \\
  Schenk\_AFE/af\_shell3          & 768                              & 0.546                      & 384  & 0.855     & 256 & \bf 0.424  \\
  DNVS/troll                      & 256                              & 0.433                      & 384  & 0.577     & 256 & \bf 0.377  \\
  AMD/G3\_circuit                 & 512                              & 2.631                      & 512  & 3.345     & 256 & \bf 0.586  \\
  GHS\_psdef/bmwcra\_1            & 256                              & 0.367                      & 384  & 0.408     & 256 & \bf 0.342  \\
  DNVS/halfb                      & 256                              & 0.437                      & 512  & 0.590     & 256 & \bf 0.372  \\
  Um/2cubes\_sphere               & 384                              & 0.476                      & 384  & 0.537     & 256 & \bf 0.323  \\
  GHS\_psdef/ldoor                & 384                              & 1.095                      & 768  & 1.618     & 256 & \bf 0.599  \\
  DNVS/ship\_003                  & 256                              & 0.405                      & 384  & 0.456     & 256 & \bf 0.364  \\
  DNVS/fullb                      & 384                              & 0.500                      & 512  & 0.603     & 256 & \bf 0.431  \\
  GHS\_psdef/inline\_1            & 384                              & 0.683                      & 512  & 0.927     & 256 & \bf 0.667  \\
  Chen/pkustk14                   & 384                              & 0.613                      & 384  & 0.635     & 256 & \bf 0.555  \\
  GHS\_psdef/apache2              & 512                              & 1.416                      & 512  & 1.848     & 256 & \bf 0.717  \\
  Koutsovasilis/F1                & 384                              & 0.812                      & 512  & 0.920     & 256 & \bf 0.786  \\
  Oberwolfach/boneS10             & 384                              & 1.186                      & 384  & 1.599     & 256 & \bf 1.111  \\
  ND/nd12k                        & 384                              & 1.478                      & 384  & \bf 1.405 & 384 & 1.498      \\
  JGD\_Trefethen/Trefethen\_20000 & 512                              & 3.692                      & 384  & \bf 2.406 & 512 & 3.829      \\
  ND/nd24k                        & 384                              & 5.379                      & 384  & \bf 5.076 & 384 & 5.498      \\
  Oberwolfach/bone010             & 384                              & 7.416                      & 768  & 7.392     & 384 & \bf 7.195  \\
  GHS\_psdef/audikw\_1            & 768                              & 10.650                     & 768  & 10.680    & 384 & \bf 10.642 \\
  \hline
\end{tabular}

    \end{center}
    \caption{Factorization times (second) obtained with MA87 and SpLLT
      (i.e. MA87\_starpu). The factorizations were run with the block
      sizes \texttt{nb=(256, 384, 512, 768, 1024)} on 28 cores and
      \texttt{nemin=32}. The lowest factorization times are
      represented in bold.}
\end{table}

%% \newpage

%% \begin{table}[htbp]
%%     \begin{center}
%%       \begin{tabular}{l|rr|r}
  \hline
  Matrix                                 & \multicolumn{2}{c}{spLLT} & MA87                   \\
  \cline{2-4}
                                         & insert (s)                & facto (s) & facto (s)  \\
  \hline
                         Schmid/thermal2 & 28.310                    & 29.180    & \bf 0.395  \\
                        Rothberg/gearbox & 0.557                     & 1.013     & \bf 0.223  \\
                              DNVS/m\_t1 & 0.347                     & 0.775     & \bf 0.222  \\
                             Boeing/pwtk & 0.944                     & 1.490     & \bf 0.255  \\
                           Chen/pkustk13 & 0.331                     & 0.757     & \bf 0.267  \\
                  GHS\_psdef/crankseg\_1 & 0.257                     & 0.729     & \bf 0.266  \\
                           Rothberg/cfd2 & 0.444                     & 0.826     & \bf 0.281  \\
                             DNVS/thread & 0.128                     & 0.536     & \bf 0.277  \\
                           DNVS/shipsec8 & 0.429                     & 0.860     & \bf 0.289  \\
                           DNVS/shipsec1 & 0.637                     & 1.056     & \bf 0.294  \\
                  GHS\_psdef/crankseg\_2 & 0.235                     & 0.906     & \bf 0.335  \\
                            DNVS/fcondp2 & 0.920                     & 1.490     & \bf 0.339  \\
                  Schenk\_AFE/af\_shell3 & 4.234                     & 5.106     & \bf 0.457  \\
                              DNVS/troll & 0.993                     & 1.746     & \bf 0.469  \\
                         AMD/G3\_circuit & 49.290                    & 50.510    & \bf 0.746  \\
                    GHS\_psdef/bmwcra\_1 & 0.709                     & 1.298     & \bf 0.400  \\
                              DNVS/halfb & 1.135                     & 1.840     & \bf 0.489  \\
                       Um/2cubes\_sphere & 0.646                     & 1.071     & \bf 0.433  \\
                        GHS\_psdef/ldoor & 14.630                    & 16.480    & \bf 0.729  \\
                          DNVS/ship\_003 & 0.593                     & 1.175     & \bf 0.470  \\
                              DNVS/fullb & 1.027                     & 1.761     & \bf 0.582  \\
                    GHS\_psdef/inline\_1 & 3.748                     & 5.462     & \bf 0.816  \\
                           Chen/pkustk14 & 0.844                     & 1.709     & \bf 0.735  \\
                      GHS\_psdef/apache2 & 12.060                    & 13.020    & \bf 0.947  \\
                        Koutsovasilis/F1 & 2.301                     & 3.888     & \bf 1.028  \\
                     Oberwolfach/boneS10 & 11.330                    & 13.610    & \bf 1.348  \\
                                ND/nd12k & 1.452                     & \bf 2.634 & 2.674      \\
         JGD\_Trefethen/Trefethen\_20000 & 3.285                     & \bf 4.844 & 22.324     \\
                                ND/nd24k & 5.875                     & \bf 8.733 & 11.945     \\
                     Oberwolfach/bone010 & 13.500                    & 18.890    & \bf 8.250  \\
                    GHS\_psdef/audikw\_1 & 16.120                    & 23.390    & \bf 12.505 \\
  \hline
\end{tabular}

%%     \end{center}
%%     \caption{Factorization times (second) obtained with MA87 and SpLLT
%%       (i.e. MA87\_starpu). The factorizations were run with the block
%%       sizes \texttt{nb=(256, 384, 512, 768, 1024)} on 28 cores and
%%       \texttt{nemin=32}. The lowest factorization times are
%%       represented in bold. This table include the time spent
%%       submitting all the tasks in the DAG.}
%% \end{table}

%% \newpage

%% \begin{table}[htbp]
%%     \begin{center}
%%       \input{table_cmp_facto_nosub.tex}
%%     \end{center}
%%     \caption{Factorization times (second) obtained with MA87 and SpLLT
%%       (i.e. MA87\_starpu). The factorizations were run with the block
%%       sizes \texttt{nb=(256, 384, 512, 768, 1024)} on 28 cores and
%%       \texttt{nemin=32}. In the case of SpLLT, the factorization start
%%       only after the whole DAG has been subbmitted to StarPU. The
%%       lowest factorization times are represented in bold. This table
%%       include the time spent submitting all the tasks in the DAG.}
%% \end{table}

%% \begin{table}[htbp]
%%     \begin{center}
%%       \begin{tabular}{l|rr|rr}
  \hline
  Matrix                          & \multicolumn{2}{c}{spLLT} & \multicolumn{2}{c}{MA87}     \\
  \cline{2-5}
                                  & nb                        & facto (s)  & nb  & facto (s) \\
  \hline
  GHS\_psdef/apache2              & 1024                      & 3.997      & 256 & \bf 1.090 \\
  Koutsovasili/F1                 & 768                       & 1.531      & 256 & \bf 1.270 \\
  Oberwolfach/boneS10             & 768                       & 2.965      & 384 & \bf 1.696 \\
  ND/nd12k                        & 512                       & \bf 2.009  & 384 & 2.923     \\
  JGD\_Trefethen/Trefethen\_20000 & 384                       & \bf 3.958  & 384 & 5.846     \\
  ND/nd24k                        & 384                       & \bf 6.740  & 384 & 11.307    \\
  Oberwolfach/bone010             & 384                       & \bf 10.250 & 768 & 14.677    \\
  GHS\_psdef/audikw\_1            & 1024                      & \bf 14.860 & 768 & 21.689    \\
  \hline
\end{tabular}

%%     \end{center}
%%     \caption{Factorization times (second) obtained with MA87 and SpLLT
%%       (i.e. MA87\_starpu). The factorizations were run with the block
%%       sizes \texttt{nb=(256, 384, 512, 768, 1024)} on 24 cores and
%%       \texttt{nemin=32}. The lowest factorization times are
%%       represented in bold.}
%% \end{table}

\end{document}
