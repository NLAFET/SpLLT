\documentclass[10pt, compress]{beamer}

\usetheme{qrmMMXV}

\usepackage{gnuplottex}
\usepackage{tikz,pgfplots}
\usepackage{booktabs}
\usepackage[scale=2]{ccicons}



\title{A new beamer theme for \texttt{qr\_mumps}}
\subtitle{}
\date{\today}
\author{Alfredo Buttari}
\institute{Institute or miscellaneous information}

\begin{document}

\maketitle

\begin{frame}[fragile]
  \frametitle{mtheme}

  The \emph{mtheme} is a Beamer theme with minimal visual noise inspired by the
  \href{https://github.com/hsrmbeamertheme/hsrmbeamertheme}{\textsc{hsrm} Beamer
  Theme} by Benjamin Weiss.

  Enable the theme by loading


  \begin{block}{pigliatela}
    nderculo
  \end{block}

  \begin{exampleblock}{pigliatela}
    nderculo
  \end{exampleblock}

  Note, that you have to have Mozilla's \emph{Fira Sans} font and XeTeX
  installed to enjoy this wonderful typography.
\end{frame}

\begin{frame}[fragile]
  \frametitle{Sections}
  Sections group slides of the same topic

\begin{verbatim}
    \section{Elements}
\end{verbatim}

  for which the \emph{mtheme} provides a nice progress indicator \ldots
\end{frame}

\section{Elements}

\begin{frame}[fragile]
  \frametitle{Typography}

The theme provides sensible defaults to \emph{emphasize}
text, \alert{accent} parts or show \textbf{bold} results.


  \begin{center}becomes\end{center}

  The theme provides sensible defaults to \emph{emphasize} text,
  \alert{accent} parts or show \textbf{bold} results.
\end{frame}
\begin{frame}{Lists}
  \begin{columns}[onlytextwidth]
    \column{0.5\textwidth}
      Items
      \begin{itemize}
        \item Milk \item Eggs \item Potatos
      \end{itemize}

    \column{0.5\textwidth}
      Enumerations
      \begin{enumerate}
        \item First, \item Second and \item Last.
      \end{enumerate}
  \end{columns}
\end{frame}
\begin{frame}{Descriptions}
  \begin{description}
    \item[PowerPoint] Meeh.
    \item[Beamer] Yeeeha.
  \end{description}
\end{frame}
\begin{frame}{Animation}
  \begin{itemize}[<+- | alert@+>]
    \item \alert<4>{This is\only<4>{ really} important}
    \item Now this
    \item And now this
  \end{itemize}
\end{frame}


\begin{frame}{Tables}
  \begin{table}
    \caption{Largest cities in the world (source: Wikipedia)}
    \begin{tabular}{lr}
      \toprule
      City & Population\\
      \midrule
      Mexico City & 20,116,842\\
      Shanghai & 19,210,000\\
      Peking & 15,796,450\\
      Istanbul & 14,160,467\\
      \bottomrule
    \end{tabular}
  \end{table}
\end{frame}
\begin{frame}{Blocks}

  \begin{block}{This is a block title}
    This is soothing.
  \end{block}

\end{frame}
\begin{frame}{Math}
  \begin{equation*}
    e = \lim_{n\to \infty} \left(1 + \frac{1}{n}\right)^n
  \end{equation*}
\end{frame}

% \begin{frame}{Line plots}
  % \begin{figure}
    % \begin{tikzpicture}
      % \begin{axis}[
        % mlineplot,
        % width=0.9\textwidth,
        % height=6cm,
      % ]

        % \addplot {sin(deg(x))};
        % \addplot+[samples=100] {sin(deg(2*x))};

      % \end{axis}
    % \end{tikzpicture}
  % \end{figure}
% \end{frame}


% \begin{frame}[fragile]{Bar charts}
% aaa bbb
% \begin{center}
% \begin{gnuplot}[terminal=pdf, scale=0.7]
% set style line 1 lt 1 lc rgb '#0072bd' # blue
% set style line 2 lt 1 lc rgb '#d95319' # orange
% set style line 3 lt 1 lc rgb '#edb120' # yellow
% set style line 4 lt 1 lc rgb '#7e2f8e' # purple
% set style line 5 lt 1 lc rgb '#77ac30' # green
% set style line 6 lt 1 lc rgb '#4dbeee' # light-blue
% set style line 7 lt 1 lc rgb '#a2142f' # red

% # Axes
% set style line 101 lc rgb '#808080' lt 1
% set border 3 back ls 101
% set tics nomirror out scale 0.75
% # Grid
% set style line 102 lc rgb'#808080' lt 0 lw 1
% set grid back ls 102

% set xrange [0:15]
% unset key

% # Bessel functions (after Bronstein 2001, eq. 9.54a)
% besj2(x) = 2*1/x * besj1(x) - besj0(x)
% besj3(x) = 2*2/x * besj2(x) - besj1(x)
% besj4(x) = 2*3/x * besj3(x) - besj2(x)
% besj5(x) = 2*4/x * besj4(x) - besj3(x)
% besj6(x) = 2*5/x * besj5(x) - besj4(x)
% besj0_(x) = x<5 ? besj0(x) : 1/0

% set label 'J_0' at 1.4,0.90 center tc ls 1
% set label 'J_1' at 1.9,0.67 center tc ls 2
% set label 'J_2' at 3.2,0.57 center tc ls 3
% set label 'J_3' at 4.3,0.51 center tc ls 4
% set label 'J_4' at 5.4,0.48 center tc ls 5
% set label 'J_5' at 6.5,0.45 center tc ls 6
% set label 'J_6' at 7.6,0.43 center tc ls 7

% plot besj0(x) ls 1 lw 2, \
%      besj1(x) ls 2 lw 2, \
%      besj2(x) ls 3 lw 2, \
%      besj3(x) ls 4 lw 2, \
%      besj4(x) ls 5 lw 2, \
%      besj5(x) ls 6 lw 2, \
%      besj6(x) ls 7 lw 2
% \end{gnuplot}
% \end{center}

% \end{frame}



\begin{frame}[fragile]{Code}

  I love Fortran!
  
  \vspace{0.5cm}

\begin{lstlisting}[basicstyle=\tt\scriptsize, showlines=true]
program pippo

  integer, parameter            :: m=10, n=5
  real(kind(1.d0)), allocatable :: a(:,:)

  allocate(a(m,n))      ! allocate the matrix

  call random_number(a) ! fill it up

  call do_something(a)  ! do something on it

  write(*,'("Hello world")')
  stop

end program pippo
\end{lstlisting}

\end{frame}

\section{Conclusion}

\begin{frame}{Summary}

  Get the source of this theme and the demo presentation from

  \begin{center}\url{github.com/matze/mtheme}\end{center}

  The theme \emph{itself} is licensed under a
  \href{http://creativecommons.org/licenses/by-sa/4.0/}{Creative Commons
  Attribution-ShareAlike 4.0 International License}.

  \begin{center}\ccbysa\end{center}

\end{frame}

\plain{Questions?}

\end{document}

%%% Local Variables:
%%% mode: latex
%%% TeX-master: t
%%% End:
